\documentclass{article}

% Basic document packages.
\usepackage{amsmath,amssymb,amsthm, amsopn}
\usepackage[margin=1.5in]{geometry}

% Font and spacing.
\usepackage{mathpazo}
\usepackage{setspace}
\setstretch{1.25}

% References. I liked the purple color.
\usepackage{xcolor}
\usepackage{hyperref}
\hypersetup{
    colorlinks=true,
    linkcolor=purple
}

% Better arrows for vector notation.
\usepackage{esvect}

% Shortcuts for sets.
\newcommand\N{\mathbb{N}}
\newcommand\Z{\mathbb{Z}}
\newcommand\Q{\mathbb{Q}}
\newcommand\R{\mathbb{R}}
\DeclareMathOperator{\Diag}{Diag}
\DeclareMathOperator{\Diagel}{Diag_{el}}
\DeclareMathOperator{\Th}{Th}
\DeclareMathOperator{\Tr}{Tr}
\DeclareMathOperator{\RM}{RM}
\let\mc\mathcal

% Basic environments for theorems et al.
\usepackage{thmtools}
\declaretheoremstyle[bodyfont=\normalfont, notefont=\normalfont\itshape]{std}
\declaretheorem[style=std]{proposition}
\declaretheorem[style=std]{definition}
\declaretheorem[style=std]{notation}
\declaretheorem[style=std]{lemma}
\declaretheorem[style=std]{theorem}
\declaretheorem[style=std]{remark}
\declaretheorem[style=std, numbered=no]{claim}

% Enumeration settings.
\usepackage{enumitem}
\setlist[enumerate,1]{
  label= (\upshape\roman*),
  ref= (\upshape\roman*),
  topsep= 1ex,
  parsep= 0pt,
}

% Boxing text in case you like that.
\usepackage[]{mdframed}
\mdfsetup{linewidth=0.9pt}

% Header for every page.
\usepackage{fancyhdr}
\pagestyle{fancy}
\setlength{\headheight}{15pt}
\renewcommand{\headrulewidth}{0.9pt}

% I hate how paragraphs work in TeX by default.
\usepackage{parskip}
\setlength{\parindent}{15pt}

\begin{document}

\chead{\textbf{Michael Fiore, \'Alvaro Ramos}}

\begin{notation}[Preliminaries] Let $\mc{L}$ be a countable language. We will also let $\mc{A}$ and $\mc{B}$ denote $\mc{L}$-structures with universes $A$ and $B$ respectively.
    \begin{itemize}
        \item Let $S$ be the set of all $\mc{L}$-formulas. Throughout this write-up, we will let $\Delta$ be a set of $\mc{L}$-formulas, so $\Delta \subseteq S$.
        \item Let $S_n$ be the set of all $\mc{L}$-formulas with at most one free variable $x$.
        \item For any set $X$, let $\mc{L}_X = \mc{L} \cup \{ c_x : x \in X \}$, where we augment $\mc{L}$ with constant symbols for every element of $X$.
        \item Let $S(X)$ be the set of all $\mc{L}_X$ formulas.
        \item Let $S_1(X)$ be defined similarly.
        \item If $X \subseteq A$, we define $\Th(\mc{A}, X)$ to be the theory of $\mc{A}$ interpreted as an $\mc{L}_X$-structure. In particular, $\Th(\mc{A}, A)$ is the elementary diagram of $\mc{A}$.
        \item An \textit{$\mc{A}$-instance} of a formula $\varphi$ is any formula obtained from $\varphi$ by substituting $c_a \in \mc{L}_A$ for each variable.
    \end{itemize}
\end{notation}

\begin{definition}[Boolean closure]
    Let $\Delta$ be a set of $\mc{L}$-formulas and let $X$ be a set. We define $\Delta^b(X)$ to be the set of formulas in $S_1(X)$ obtained from formulas in $\Delta$ using conjunction, disjunction, negation, and substituting instances of $c_x$ for a variable. Trivially, $S^b(A) = S(A)$.
\end{definition}

\begin{definition}[Partitioning]
   Let $\varphi \in S_1(B)$ and $\Gamma = \{ \psi_1, \ldots, \psi_n \} \subset S_1(B)$. We say that $\Gamma$ \textit{partitions} $\varphi$ if we have the following.
   \begin{alignat*}{3}
        \mc{B} &\vDash \forall x (\psi_1(x) \lor \cdots \lor \psi_n(x) \lor \neg \varphi(x)) \\
        \mc{B} &\vDash \forall x ((\psi_i(x) \land \psi_j(x)) \to \neg \varphi(x)) &\text{ for all } i \neq j \\
        \mc{B} &\vDash \exists x (\varphi(x) \land \psi_i(x)) &\text{ for all } i
   \end{alignat*}
   In other words, whenever $\varphi(x)$ is true in $\mc{B}$, exactly one of the $\psi_i(x)$'s is true.
\end{definition}

\begin{definition}[Generalized rank and degree]
    Let $\Delta \subseteq A$. In order to define a generalized version of Morley rank and degree, we will need to define two sets $S^{\alpha}(\mc{A}, \Delta)$ and $\Tr^{\alpha}(\mc{A}, \Delta)$ for any ordinal $\alpha$ inductively.
    \begin{itemize}
        \item Let $S^{0}(\mc{A}, \Delta) = \{ \varphi \in S_1(A) : \mc{A} \vDash \exists x \varphi(x) \}$, i.e. the set of all $\mc{L}$-formulas that have witnesses in $\mc{A}$.

        \item If $S^{\alpha}(\mc{A}, \Delta)$ is already defined, we may define $\Tr^{\alpha}(\mc{A}, \Delta)$ as follows. Given a formula $\varphi$, we say that $\varphi \in \Tr^{\alpha}(\mc{A}, \Delta)$ if and only if there exists a finite $k$ such that for all $\mc{B} \succeq \mc{A}$ and any finite $\Gamma \subset \Delta^{b}(B)$ partitioning $\mc{A}$, there are no more than $k$ formulas $\psi$ in $\Gamma$ such that $(\varphi \land \psi) \in S^{\alpha}(\mc{A}, \Delta)$.
        
        \item Let $S^{\alpha+1}(\mc{A}, \Delta) = S^{\alpha}(\mc{A}, \Delta) \setminus \Tr^{\alpha}(\mc{A}, \Delta)$.

        \item For $\alpha$ limit, let $S^{\alpha}(\mc{A}, \Delta) = \bigcap_{\beta < \alpha} S^{\beta}(\mc{A}, \Delta)$.

        \item The \textit{$\Delta$-rank} of $\varphi$ in $\mc{A}$ is the least ordinal $\alpha$ such that $\varphi \in \Tr^{\alpha}(\mc{A}, \Delta)$.
        
        \item The \textit{$\Delta$-degree} of $\varphi$ in $\mc{A}$ is the least number $k$ witnessing that $\varphi \in \Tr^{\alpha}(\mc{A}, \Delta)$.
    \end{itemize}
\end{definition}

\begin{proposition}[$\Delta$-rank generalizes Morley rank]\label{dRank}
    Let $p$ be a 1-type in $\Th(\mc{A}, A)$. The Morley rank of $p$ is the least $p$ such that $p \cap \Tr^{\alpha}(\mc{A}, S) \neq \varnothing$. The Morley degree of $p$ is the minimum of the $S$-degrees of the formulas in $p \cap \Tr^{\alpha}(\mc{A}, S)$.
\end{proposition}
\begin{proof}
\end{proof}

\begin{definition}[Minimality]
    A formula $\varphi$ is \textit{minimal} in $\Tr^{\alpha}(\mc{A}, \Delta)$ if $\varphi \in \Tr^{\alpha}(\mc{A}, \Delta)$ and there is no $\psi \in \Delta$ with some $\mc{A}$-instance $\psi^{\prime}$ such that $(\varphi \land \psi^{\prime})$ and $(\varphi \land \neg \psi^{\prime})$ are both in $\Tr^{\alpha}(\mc{A}, \Delta)$.
\end{definition}

\begin{lemma}[Properties of rank]\label{pRank}
    Let $\varphi$, $\varphi_0$, $\varphi_1$, and $\psi$ be formulas in $S_1(A)$.
    \begin{enumerate}
        \item If $\varphi(\mc{A}) \subseteq \psi(\mc{A})$ and $\varphi \in S^{\alpha}(\mc{A}, \Delta)$, then $\psi \in S^{\alpha}(\mc{A}, \Delta)$.
        \item Suppose $\mc{B} \succeq \mc{A}$. Then $\varphi \in S^{\alpha}(\mc{A}, \Delta)$ if and only if $\varphi \in S^{\alpha}(\mc{B}, \Delta)$.
        \item If $(\varphi_0 \lor \varphi_1)(\mc{A}) \supseteq \varphi(\mc{A})$ and $\varphi \in S^{\alpha}(\mc{A}, \Delta)$, then one of $\varphi_0$ and $\varphi_1$ is in $S^{\alpha}(\mc{A}, \Delta)$.
    \end{enumerate}
\end{lemma}
\begin{proof}
\end{proof}

\begin{definition}[Weak satisfiability]
    First, for any set of formulas $\Theta$, we define
    \[ \Theta^{-} = \{ \neg \phi : \phi \in \Theta \} \]
    to be the set of negations of all formulas in $\Theta$. The set $\Gamma \subset S(A)$ is \textit{weakly satisfiable} in $\mc{A}$ if no finite disjunctions of formulas in $\Gamma^{-}$ is valid in $\mc{A}$.
\end{definition}

\begin{remark}\label{wSat}
    A set $\Gamma \subset S(A)$ is weakly satisfiable in $\mc{A}$ if and only if $\Gamma$ is satisfiable in some $\mc{B} \succeq \mc{A}$.
\end{remark}
\begin{proof}
\end{proof}

\begin{lemma}[A characterization of rank]\label{gStar}
    Let $\mc{A}$ and $\mc{B}$ be $\mc{L}$-structures. Let $\varphi \in S^{\alpha}(\mc{A}, \Delta)$. Let $\Gamma$ be the set of all formulas obtained by switching around variables in the formulas of $\{ \phi \} \cup \Delta \cup \Delta^{-}$. There exists $\Gamma^{\ast} \subseteq \Gamma$ weakly satisfiable in $\mc{A}$ such that if $\psi \in S_1(B)$ and $\Gamma^{\dagger}$ is weakly satisfiable in $\mc{B}$, where $\Gamma^{\dagger}$ is obtained from $\Gamma^{\ast}$ by replacing each instance of $\varphi$ by the corresponding instance of $\psi$, then $\psi \in S^{\alpha}(\mc{B}, \Delta)$.
\end{lemma}
\begin{proof}
\end{proof}

\begin{lemma}\label{dFinite}
    Let $n \in \omega$, let $\Delta$ be finite, and $\varphi$ be an $\mc{L}_X$-formula containing at most one variable $x$ free and possibly names for the elements of the universe of some model. There exists $\Gamma^{*} \subseteq S(X)$ depending only on $n$, $\varphi$, and $\Delta$ such that for any $\mc{A}$, if $\varphi \in S_1(A)$, then $\varphi \in S^n(\mc{A}, \Delta)$ if and only if $\Gamma^{*}$ is weakly satisfied in $\mc{A}$.
\end{lemma}
\begin{proof}
\end{proof}

\begin{lemma}
    Let $\Delta$ be finite, $n \in \omega$, and let $\varphi \in \Tr^{n}(\mc{A}, \Delta)$ have $\Delta$-degree 1. For each $\psi \in (\Delta \cup \Delta^{-}) \cap S_{k+1}$ there exists $\theta \in S_k(A)$ such that if $\mc{B} \succeq \mc{A}$, then 
    \[ (\varphi \land \psi)(x, b_1, \ldots, b_k) \in \Tr^{n}(\mc{B}, \Delta) \Leftrightarrow \mc{B} \vDash \theta(b_1, \ldots, b_k), \]
    where $b_1, \ldots, b_k$ are arbitrary in $B$.
\end{lemma}
\begin{proof}
\end{proof}

\begin{lemma}
    If $\Delta$ is finite then $\Tr^{\alpha}(\mc{A}, \Delta) = \varnothing$ if $\alpha \geq \omega$.
\end{lemma}
\begin{proof}
\end{proof}

\begin{definition}[$\Delta$-stable]
\end{definition}

\begin{lemma}
    Let $\mc{A} \preceq \mc{B}$, let $\varphi$ be a formula in $S_1(A)$ which is $\Delta$-stable, and let $\psi \in \Delta^b(B)$. Then there exists $\theta \in S_1(A)$ such that $\varphi(\mc{A}) \cap \psi(\mc{B}) = \theta(\mc{A})$.
\end{lemma}
\begin{proof}
    No proof but some references listed, including Shelah.
\end{proof}

\begin{lemma}
    If $\varphi$ is minimal in $\Tr^{\alpha}(\mc{A}, S)$, then the $S$-degree of $\varphi$ is 1.
\end{lemma}
\begin{proof}
\end{proof}

\begin{lemma}
    If $\Delta$ is finite and $\varphi$ is minimal in $\Tr^n(\mc{A}, \Delta)$, then the $\Delta$-degree of $\varphi$ in $\mc{A}$ is 1.
\end{lemma}
\begin{proof}
\end{proof}

\begin{theorem}[Model extension]
    Let $\mc{A}$ and $\mc{B}$ be models of a countable stable theory and suppose that $\mc{A} \prec \mc{B}$ and $P(\mc{A}) = P(\mc{B})$ where $P$ is a unary predicate symbol. There exists $\mc{C} \succ \mc{B}$ such that $P(\mc{C}) = P(\mc{B})$.
\end{theorem}
\begin{proof}
\end{proof}

\end{document}