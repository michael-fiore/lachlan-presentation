\documentclass{article}

% Basic document packages.
\usepackage{amsmath,amssymb,amsthm, amsopn}
\usepackage[margin=1.5in]{geometry}

% Font and spacing.
\usepackage{mathpazo}
\usepackage{setspace}
\setstretch{1.25}

% References. I liked the purple color.
\usepackage{xcolor}
\usepackage{hyperref}
\hypersetup{
    colorlinks=true,
    linkcolor=purple
}

% Better arrows for vector notation.
\usepackage{esvect}

% Shortcuts for sets.
\newcommand\N{\mathbb{N}}
\newcommand\Z{\mathbb{Z}}
\newcommand\Q{\mathbb{Q}}
\newcommand\R{\mathbb{R}}
\DeclareMathOperator{\Diag}{Diag}
\DeclareMathOperator{\Diagel}{Diag_{el}}
\DeclareMathOperator{\Th}{Th}
\DeclareMathOperator{\Tr}{Tr}
\DeclareMathOperator{\RM}{RM}
\let\mc\mathcal

% Basic environments for theorems et al.
\usepackage{thmtools}
\declaretheoremstyle[bodyfont=\normalfont, notefont=\normalfont\itshape]{std}
\declaretheorem[style=std]{proposition}
\declaretheorem[style=std]{definition}
\declaretheorem[style=std]{notation}
\declaretheorem[style=std]{lemma}
\declaretheorem[style=std]{theorem}
\declaretheorem[style=std]{remark}
\declaretheorem[style=std, numbered=no]{claim}

% Enumeration settings.
\usepackage{enumitem}
\setlist[enumerate,1]{
  label= (\upshape\roman*),
  ref= (\upshape\roman*),
  topsep= 1ex,
  parsep= 0pt,
}

% Boxing text in case you like that.
\usepackage[]{mdframed}
\mdfsetup{linewidth=0.9pt}

% Header for every page.
\usepackage{fancyhdr}
\pagestyle{fancy}
\setlength{\headheight}{15pt}
\renewcommand{\headrulewidth}{0.9pt}

% I hate how paragraphs work in TeX by default.
\usepackage{parskip}
\setlength{\parindent}{15pt}

\begin{document}

\chead{\textbf{Michael Fiore, \'Alvaro Ramos}}

\begin{notation}[Preliminaries] Let $\mc{L}$ be a countable language. We will also let $\mc{A}$ and $\mc{B}$ denote $\mc{L}$-structures with universes $A$ and $B$ respectively.
    \begin{itemize}
        \item Let $S$ be the set of all $\mc{L}$-formulas. Throughout this write-up, we will let $\Delta$ be a set of $\mc{L}$-formulas, so $\Delta \subseteq S$.
        \item Let $S_n$ be the set of all $\mc{L}$-formulas with at most one free variable $x$.
        \item For any set $X$, let $\mc{L}_X = \mc{L} \cup \{ c_x : x \in X \}$, where we augment $\mc{L}$ with constant symbols for every element of $X$.
        \item Let $S(X)$ be the set of all $\mc{L}_X$ formulas.
        \item Let $S_1(X)$ be defined similarly.
        \item If $X \subseteq A$, we define $\Th(\mc{A}, X)$ to be the theory of $\mc{A}$ interpreted as an $\mc{L}_X$-structure. In particular, $\Th(\mc{A}, A)$ is the elementary diagram of $\mc{A}$.
        \item An \textit{$\mc{A}$-instance} of a formula $\varphi$ is any formula obtained from $\varphi$ by substituting $c_a \in \mc{L}_A$ for each variable.
    \end{itemize}
\end{notation}

\begin{definition}[Boolean closure]
    Let $\Delta$ be a set of $\mc{L}$-formulas and let $X$ be a set. We define $\Delta^b(X)$ to be the set of formulas in $S_1(X)$ obtained from formulas in $\Delta$ using conjunction, disjunction, negation, and substituting instances of $c_x$ for a variable. Trivially, $S^b(A) = S(A)$.
\end{definition}

\begin{definition}[Partitioning]
   Let $\varphi \in S_1(B)$ and $\Gamma = \{ \psi_1, \ldots, \psi_n \} \subset S_1(B)$. We say that $\Gamma$ \textit{partitions} $\varphi$ if we have the following.
   \begin{alignat*}{3}
        \mc{B} &\vDash \forall x (\psi_1(x) \lor \cdots \lor \psi_n(x) \lor \neg \varphi(x)) \\
        \mc{B} &\vDash \forall x ((\psi_i(x) \land \psi_j(x)) \to \neg \varphi(x)) &\text{ for all } i \neq j \\
        \mc{B} &\vDash \exists x (\varphi(x) \land \psi_i(x)) &\text{ for all } i
   \end{alignat*}
   In other words, whenever $\varphi(x)$ is true in $\mc{B}$, exactly one of the $\psi_i(x)$'s is true.
\end{definition}

\begin{definition}[Generalized rank and degree]
    Let $\Delta \subseteq A$. In order to define a generalized version of Morley rank and degree, we will need to define two sets $S^{\alpha}(\mc{A}, \Delta)$ and $\Tr^{\alpha}(\mc{A}, \Delta)$ for any ordinal $\alpha$ inductively.
    \begin{itemize}
        \item Let $S^{0}(\mc{A}, \Delta) = \{ \varphi \in S_1(A) : \mc{A} \vDash \exists x \varphi(x) \}$, i.e. the set of all $\mc{L}$-formulas that have witnesses in $\mc{A}$.

        \item If $S^{\alpha}(\mc{A}, \Delta)$ is already defined, we may define $\Tr^{\alpha}(\mc{A}, \Delta)$ as follows. Given a formula $\varphi$, we say that $\varphi \in \Tr^{\alpha}(\mc{A}, \Delta)$ if and only if there exists a finite $k$ such that for all $\mc{B} \succeq \mc{A}$ and any finite $\Gamma \subset \Delta^{b}(B)$ partitioning $\mc{A}$, there are no more than $k$ formulas $\psi$ in $\Gamma$ such that $(\varphi \land \psi) \in S^{\alpha}(\mc{A}, \Delta)$.
        
        \item Let $S^{\alpha+1}(\mc{A}, \Delta) = S^{\alpha}(\mc{A}, \Delta) \setminus \Tr^{\alpha}(\mc{A}, \Delta)$.

        \item For $\alpha$ limit, let $S^{\alpha}(\mc{A}, \Delta) = \bigcap_{\beta < \alpha} S^{\beta}(\mc{A}, \Delta)$.

        \item The \textit{$\Delta$-rank} of $\varphi$ in $\mc{A}$ is the least ordinal $\alpha$ such that $\varphi \in \Tr^{\alpha}(\mc{A}, \Delta)$.
        
        \item The \textit{$\Delta$-degree} of $\varphi$ in $\mc{A}$ is the least number $k$ witnessing that $\varphi \in \Tr^{\alpha}(\mc{A}, \Delta)$.
    \end{itemize}
\end{definition}

\begin{proposition}[$\Delta$-rank generalizes Morley rank]\label{dRank}
    Let $p$ be a 1-type in $\Th(\mc{A}, A)$. The Morley rank of $p$ is the least $p$ such that $p \cap \Tr^{\alpha}(\mc{A}, S) \neq \varnothing$. The Morley degree of $p$ is the minimum of the $S$-degrees of the formulas in $p \cap \Tr^{\alpha}(\mc{A}, S)$.
\end{proposition}

\begin{proof}
    Let $\mc{B} \succeq \mc{A}$ be an $\aleph_0$-saturated model, so $p$ is realized in $\mc{B}$. To prove the first part, we show by induction on $\alpha$ that for all $\varphi \in p$,
    \[ \RM(\varphi) = \RM^{\mc{B}}(\varphi) \geq \alpha \Leftrightarrow \varphi \in S^{\alpha}(\mc{A}, S). \]
    If $\alpha = 0$, we clearly have by definition that $\mc{A} \vDash \exists x \varphi(x)$ if and only if $\varphi(\mc{A})$ is nonempty. The limit case follows easily by induction.

    As for the successor case, we need to show that $\RM(\varphi) \geq \alpha + 1$ if and only if $\phi$ is in $S^{\alpha}(\mc{A}, S) \setminus \Tr^{\alpha}(\mc{A}, S)$. First suppose that $\RM(\varphi) \geq \alpha + 1$, so there exist infinitely many $\mc{L}_B$-formulas $\psi_1, \psi_2, \ldots$ defining pairwise disjoint subsets of $\varphi(\mc{B})$ such that $\RM(\psi_i) \geq \alpha$ for all $i < \omega$. Let $k$ be any finite number. Then, define the following finite set of formulas $\Gamma \subset S^b(B)$.
    \[ \Gamma = \left\{ \psi_1, \ldots, \psi_{k+1}, \left( \varphi \land \bigwedge_{i=1}^{k+1} \neg \psi_i \right) \right\} \]
    This set clearly partitions $\varphi$. We can see that for at least $k+1$ formulas $\psi_i \in \Gamma$, $\psi_i \land \varphi = \psi_i$ has rank $\alpha$, which by induction means $\psi_i \in S^{\alpha}(\mc{B}, S)$. Thus, we conclude that $\varphi$ cannot be in $\Tr^{\alpha}(\mc{A}, S)$.

    Now suppose that $\RM(\varphi) = \alpha$. Then, we can see that the Morley degree of $\varphi$ as an $\mc{L}_B$-formula is a $k$ witnessing that $\varphi \in \Tr^{\alpha}(\mc{A}, S)$. In fact, by the definition of the Morley degree of a type, the second part of the proposition is trivial.
\end{proof}

\begin{definition}[Minimality]
    A formula $\varphi$ is \textit{minimal} in $\Tr^{\alpha}(\mc{A}, \Delta)$ if $\varphi \in \Tr^{\alpha}(\mc{A}, \Delta)$ and there is no $\psi \in \Delta$ with some $\mc{A}$-instance $\psi^{\prime}$ such that $(\varphi \land \psi^{\prime})$ and $(\varphi \land \neg \psi^{\prime})$ are both in $\Tr^{\alpha}(\mc{A}, \Delta)$.
\end{definition}

\begin{lemma}[Properties of rank]\label{pRank}
    Let $\varphi$, $\varphi_0$, $\varphi_1$, and $\psi$ be formulas in $S_1(A)$.
    \begin{enumerate}
        \item If $\varphi(\mc{A}) \subseteq \psi(\mc{A})$ and $\varphi \in S^{\alpha}(\mc{A}, \Delta)$, then $\psi \in S^{\alpha}(\mc{A}, \Delta)$.
        \item Suppose $\mc{B} \succeq \mc{A}$. Then $\varphi \in S^{\alpha}(\mc{A}, \Delta)$ if and only if $\varphi \in S^{\alpha}(\mc{B}, \Delta)$.
        \item If $(\varphi_0 \lor \varphi_1)(\mc{A}) \supseteq \varphi(\mc{A})$ and $\varphi \in S^{\alpha}(\mc{A}, \Delta)$, then one of $\varphi_0$ and $\varphi_1$ is in $S^{\alpha}(\mc{A}, \Delta)$.
    \end{enumerate}
\end{lemma}
\begin{proof}
    (i) Clearly $\varphi \in S^0(\mc{A}, \Delta)$ implies that $\psi(\mc{A}) \supseteq \varphi(\mc{A}) \neq \varnothing$, so $\psi \in S^0(\mc{A}, \Delta)$. For the successor case, given a set which partitions $\varphi$ with $k$ formulas in $S^{\alpha}(\mc{A}, \Delta)$, we easily also have a set which partitions $\psi$ with $k$ formulas in $S^{\alpha}(\mc{A}, \Delta)$. Thus, if $\varphi \notin \Tr^{\alpha}(\mc{A}, \Delta)$, then $\psi \notin \Tr^{\alpha}(\mc{A}, \Delta)$. The limit case follows easily from induction.

    (ii) Since $\mc{A} \vDash \exists x \varphi(x)$ if and only if $\mc{B} \vDash \exists x \varphi(x)$, we immediately have the case of $\alpha = 0$. In fact, since the property of a set partitioning a formula depends only on the satisfaction of finitely many sentences, a set partitioning $\varphi$ in $\mc{A}$ with $k$ formulas in $S^{\alpha}(\mc{A}, \Delta)$ will also be a set partitioning $\varphi$ in $\mc{B}$ with $k$ formulas in $S^{\alpha}(\mc{B}, \Delta)$ and vice versa by induction. The limit case also follows easily.
    
    (iii) We have the following deduction to show that this holds for the case of $\alpha = 0$.
    \[ \mc{A} \vDash \exists x \varphi (x) \Rightarrow \mc{A} \vDash \exists x (\varphi_0 (x) \lor \varphi_1(x)) \Rightarrow \mc{A} \vDash \exists x \varphi_0(x) \lor \exists x \varphi_1(x) \]
    
    Suppose the lemma holds for $\alpha$, suppose $\varphi \in S^{\alpha+1}(\mc{A}, \Delta)$, and let $k \in \omega$. There exists a set $\Gamma$ which partitions $\varphi$ with formulas $\psi_1, \ldots, \psi_{2k} \in \Gamma$ such that each $\psi_i$ is in $S^{\alpha}(\mc{A}, \Delta)$. For each $\psi_i$, at least one of $(\psi_i \land \varphi_0)$ and $(\psi_i \land \varphi_1)$ is in $S^{\alpha}(\mc{A}, \Delta)$ by induction. Since we have $2k$ formulas, for at least one of the $\varphi_j$'s there are at least $k$ formulas of the form $(\psi_i \land \varphi_j)$ in $S^{\alpha}(\mc{A}, \Delta)$. Then we can easily construct a partitioning set showing that $\varphi_j$ cannot have degree $k$. We are able to show that for any $k \in \omega$ one of $\varphi_0$ and $\varphi_1$ is unable to have degree $k$. Thus, for one of the $\varphi_j$'s, $\varphi_j$ fails to have degree $k$ for infinitely many $k$, meaning $\varphi_j \in S^{\alpha+1}(\mc{A}, \Delta)$. 
    
    Finally, for the limit case, assuming the lemma holds for all $\beta < \alpha$, it is true for at least one of the $\varphi_j$'s that there is a cofinal sequence of ordinals in $\alpha$ where for each entry $\beta$, $\varphi_j \in S^{\beta}(\mc{A}, \Delta)$. Thus, $\varphi_j \in S^{\alpha}(\mc{A}, \Delta)$.
\end{proof}

\begin{definition}[Weak satisfiability]
    First, for any set of formulas $\Theta$, we define
    \[ \Theta^{-} = \{ \neg \phi : \phi \in \Theta \} \]
    to be the set of negations of all formulas in $\Theta$. The set $\Gamma \subset S(A)$ is \textit{weakly satisfiable} in $\mc{A}$ if no finite disjunctions of formulas in $\Gamma^{-}$ is valid in $\mc{A}$.
\end{definition}

\begin{remark}\label{wSat}
    A set $\Gamma \subset S(A)$ is weakly satisfiable in $\mc{A}$ if and only if $\Gamma$ is satisfiable in some $\mc{B} \succeq \mc{A}$.
\end{remark}
\begin{proof}
    For the rightward direction, $\Gamma \cup \Diagel(\mc{A})$ is finitely satisfiable and therefore satisfiable. For the leftward direction, if $\Gamma \cup \Diagel(\mc{A})$ were not satisfiable, then there would exist formulas $\psi_1, \ldots, \psi_n \in \Gamma^{-}$ such that $\Diagel(\mc{A}) \vDash \forall \overline{x} (\psi_1 \lor \cdots \lor \psi_n)$, so $\Gamma$ would not be weakly satisfiable in $\mc{A}$.
\end{proof}

\begin{lemma}[A characterization of rank]\label{gStar}
    Let $\mc{A}$ and $\mc{B}$ be $\mc{L}$-structures. Let $\varphi \in S^{\alpha}(\mc{A}, \Delta)$. Let $\Gamma$ be the set of all formulas obtained by switching around variables in the formulas of $\{ \phi \} \cup \Delta \cup \Delta^{-}$. There exists $\Gamma^{\ast} \subseteq \Gamma$ weakly satisfiable in $\mc{A}$ such that if $\psi \in S_1(B)$ and $\Gamma^{\dagger}$ is weakly satisfiable in $\mc{B}$, where $\Gamma^{\dagger}$ is obtained from $\Gamma^{\ast}$ by replacing each instance of $\varphi$ by the corresponding instance of $\psi$, then $\psi \in S^{\alpha}(\mc{B}, \Delta)$.
\end{lemma}
\begin{proof}
    We proceed by induction on $\alpha$. For $\alpha = 0$, the set $\Gamma^{\prime} = \{ \varphi \}$ works, as well as its obvious counterpart $\Gamma^{\dagger} = \{ \psi \}$. 
    
    If $\alpha$ is a limit ordinal, suppose the lemma holds for all $\beta < \alpha$. Then, for each $\beta$, there exists some $\Gamma_{\beta}^{\ast} \subseteq \Gamma$ weakly satisfiable in $\mc{A}$ such that the corresponding set $\Gamma_{\beta}^{\dagger}$ has the desired property of the lemma. We may change the variables in each $\Gamma_{\beta}$ such that if $\beta, \gamma < \alpha$ and $\beta \neq \gamma$, then $\Gamma_{\beta}$ and $\Gamma_{\gamma}$ have no variables in common. Then, let $\Gamma^{\ast} = \bigcup_{\beta < \alpha} \Gamma_{\beta}^{\ast}$, so $\Gamma^{\dagger} = \bigcup_{\beta < \alpha} \Gamma_{\beta}^{\dagger}$. Clearly, $\Gamma^{\ast}$ is weakly satisfiable in $\mc{A}$. If $\Gamma^{\dagger}$ is weakly satisfiable in $\mc{B}$, then each $\Gamma_{\beta}^{\dagger}$ is weakly satisfiable. By induction, that means $\psi \in S^{\beta}(\mc{B}, \Delta)$ for all $\beta < \alpha$ and so $\psi \in S^{\alpha}(\mc{B}, \Delta)$.

    Now suppose that $\alpha$ is a successor with $\alpha = \beta + 1$. We will want to construct a nice sequence of formulas in order to construct $\Gamma^{\ast}$.
    \begin{claim}
    There exists $\mc{A}^{\prime} \succeq \mc{A}$ and a sequence $\langle \varphi_n : n \in \omega \rangle$ of formulas in $\Delta^b(A^{\prime})$ such that, for all $m, n \in \omega$,
    \begin{itemize}
        \item $(\varphi \land \varphi_n) \in S^{\beta}(\mc{A}^{\prime}, \Delta)$,
        \item $\neg(\varphi_m \land \varphi_n)$ is a tautology, i.e. valid in every structure, if $m \neq n$, and 
        \item $\varphi_n$ is a conjunction of $\mc{A}^{\prime}$-instances of formulas in $\Delta \cup \Delta^{-}$.
    \end{itemize}
    \end{claim}
    First, since $\varphi \in S^{\alpha}(\mc{A}, \Delta)$, there exists $\mc{A}_0 \succeq \mc{A}$ and mutually exclusive $\theta_0, \theta_1 \in \Delta^b(A_0)$ such that $(\varphi \land \theta_0), (\varphi \land \theta_1) \in S^{\beta}(\mc{A}_0, \Delta)$. By \autoref{pRank}(iii), we can let $\theta$ be one of $\theta_0$, $\theta_1$, and $(\neg\theta_0 \land \neg\theta_1)$ such that $(\varphi \land \theta) \in S^{\alpha}(\mc{A}_0, \Delta)$. If $\theta = \theta_0$, let $\varphi_0^{\ast} = \theta_1$. Otherwise, let $\varphi_0^{\ast} = \theta_0$. By repeated use of \autoref{pRank}(iii) and laws of Boolean algebra, we can find $\varphi_0$, a conjunction of $\mc{A}_0$-instances of formulas in $\Delta \cup \Delta^{-}$ such that $\varphi_0(\mc{A}_0) \subseteq \varphi_0^{\ast}(\mc{A}_0)$ and $\varphi \in S^{\beta}(\mc{A}_0, \Delta)$.

    Since $\theta \in S^{\alpha}(\mc{A}, \Delta)$, we can repeat this process on $\theta$ instead of $\varphi$ and, indeed, continue it indefinitely. This gives us a sequence of formulas $\langle \varphi_n : n \in \omega \rangle$ and an increasing chain of models
    \[ \mc{A} \preceq \mc{A}_0 \preceq \mc{A}_1 \preceq \cdots \]
    where, for all $n$, $\varphi_n$ is a conjunction of $\mc{A}_n$-instances of formulas in $\Delta \cup \Delta^{-}$. Finally, let $\mc{A}^{\prime} = \bigcup_{n \in \omega} \mc{A}_n$. Then, for all $n$, $(\varphi \land \varphi_n) \in S^{\beta}(\mc{A}^{\prime}, \Delta)$ by \autoref{pRank}(ii). The sequence also satisfies the other two desired properties, meaning we have proven the claim.

    \begin{mdframed}

        There is an issue with this proof: I have shown that these formulas are mutually exclusive in $\mc{A}^{\prime}$, but this is a weaker notion than the second part of the claim. To salvage this: make it so $\varphi_{n+1}$ has the negation of a formula $\delta_m \in \Delta \cup \Delta^{-}$ for all $m \leq n$, where $\delta_m$ appears in $\varphi_m$ but not $\varphi_{m-1}$.
    \end{mdframed}

    For each $n$, let $\Gamma_n$ be the set of all formulas obtained by switching around variables in the formulas of $\{ (\varphi \land \varphi_n) \} \cup \Delta \cup \Delta^{-}$. Since we have assumed the lemma for $\beta$, there exists $\Gamma_n^{\prime} \subseteq \Gamma_n$ weakly satisfiable in $\mc{A}^{\prime}$ with our desired property. We can also assume that $\Gamma_m$ and $\Gamma_n$ share no variables if $m \neq n$ and that there are infinitely many variables not occurring in any $\Gamma_n$.

    For each $c_a \in \mc{L}_A$ which occurs in one of our formulas $\varphi_n$ we can set aside a variable $x_a$; this variable is independent of $n$. Let $\Gamma_n^{\ast}$ be obtained from $\Gamma_n$ by replacing each occurrence of $c_a$ with a corresponding variable $x_a$. Let $\Gamma^{\ast} = \bigcup_{n \in \omega} \Gamma_n^{\ast}$. We can see that $\Gamma^{\ast}$ is weakly satisfiable in $\mc{A}$. Now, let $\Gamma^{\dagger}$ be the set obtained from $\Gamma^{\ast}$ by replacing each instance of $\varphi$ with $\psi$.
    
    Suppose $\Gamma^{\dagger}$ is weakly satisfiable in $\mc{B}$. Then, by \autoref{wSat}, there exists $\mc{B}^{\prime} \succeq \mc{B}$ where $\Gamma^{\dagger}$ is satisfied. By replacing each variable $v_a$ with a constant symbol in $\mc{L}_{B^{\prime}}$ appropriately, we have a sequence of formulas $\langle \psi_n : n \in \omega \rangle$ corresponding to each $\varphi_n$. This gives us infinitely many disjoint formulas $(\psi \land \psi_n) \in S^{\beta}(\mc{B}^{\prime}, \Delta)$, witnessing that $\psi \in S^{\alpha}(\mc{B}^{\prime}, \Delta)$. By \autoref{pRank}(ii), $\psi \in S^{\alpha}(\mc{B}, \Delta)$.
\end{proof}

\begin{lemma}\label{dFinite}
    Let $\Gamma$ be defined as in \autoref{gStar}. Let $n \in \omega$, let $\Delta$ be finite, and $\varphi$ be an $\mc{L}_X$-formula containing at most one variable $x$ free and possibly names for the elements of the universe of some model. There exists $\Gamma^{*} \subseteq S(X)$ depending only on $n$, $\varphi$, and $\Delta$ such that for any $\mc{A}$, if $\varphi \in S_1(A)$, then $\varphi \in S^n(\mc{A}, \Delta)$ if and only if $\Gamma^{*}$ is weakly satisfied in $\mc{A}$.
\end{lemma}
\begin{proof}
    We proceed by induction on $n$. For $n = 0$, $\Gamma^{\ast} = \{ \varphi(x) \}$ clearly works. Now assume that the result is true for $n = m$ where $m \geq 0$.

    For $\psi \in S^{m+1}(\mc{B}, \Delta)$, let $\Theta(\mc{B})$ be the set of all formulas $\theta \in \Delta \cup \Delta^{-}$ such that $(\psi \land \theta^{\prime}) \in S^{m+1}(\mc{B}^{\prime}, \Delta)$ and $(\psi \land \neg\theta^{\prime}) \in S^m(\mc{B}^{\prime}, \Delta)$ for some $\mc{B}^{\prime} \succeq \mc{B}$ and some $\mc{B}^{\prime}$-instance $\theta^{\prime}$ of $\theta$. Given $\varphi \in S^{m+1}(\mc{B}, \Delta)$, choose $\mc{B} \succeq \mc{A}$ and $\psi \in S^{m+1}(\mc{B}, \Delta)$ such that 
\end{proof}

\begin{lemma}
    Let $\Delta$ be finite, $n \in \omega$, and let $\varphi \in \Tr^{n}(\mc{A}, \Delta)$ have $\Delta$-degree 1. For each $\psi \in (\Delta \cup \Delta^{-}) \cap S_{k+1}$ there exists $\theta \in S_k(A)$ such that if $\mc{B} \succeq \mc{A}$, then 
    \[ (\varphi \land \psi)(x, b_1, \ldots, b_k) \in \Tr^{n}(\mc{B}, \Delta) \Leftrightarrow \mc{B} \vDash \theta(b_1, \ldots, b_k), \]
    where $b_1, \ldots, b_k$ are arbitrary in $B$.
\end{lemma}
\begin{proof}
    Short proof, looks terse.
\end{proof}

\begin{lemma}
    If $\Delta$ is finite then $\Tr^{\alpha}(\mc{A}, \Delta) = \varnothing$ if $\alpha \geq \omega$.
\end{lemma}
\begin{proof}
    No proof but explains some analogous results.
\end{proof}

\begin{definition}[$\Delta$-stable]
    To be added. Generalization of stability (duh).
\end{definition}

\begin{lemma}
    Let $\mc{A} \preceq \mc{B}$, let $\varphi$ be a formula in $S_1(A)$ which is $\Delta$-stable, and let $\psi \in \Delta^b(B)$. Then there exists $\theta \in S_1(A)$ such that $\varphi(\mc{A}) \cap \psi(\mc{B}) = \theta(\mc{A})$.
\end{lemma}
\begin{proof}
    No proof but some references listed, including Shelah.
\end{proof}

\begin{lemma}
    If $\varphi$ is minimal in $\Tr^{\alpha}(\mc{A}, S)$, then the $S$-degree of $\varphi$ is 1.
\end{lemma}
\begin{proof}
    Long, seemingly full proof given.
\end{proof}
That last lemma says that all transcendental points in the Stone space of a model have degree 1.

\begin{lemma}
    If $\Delta$ is finite and $\varphi$ is minimal in $\Tr^n(\mc{A}, \Delta)$, then the $\Delta$-degree of $\varphi$ in $\mc{A}$ is 1.
\end{lemma}
\begin{proof}
    Lachlan's experimental notation really shines here.
\end{proof}

\begin{theorem}[Model extension]
    Let $\mc{A}$ and $\mc{B}$ be models of a countable stable theory and suppose that $\mc{A} \prec \mc{B}$ and $P(\mc{A}) = P(\mc{B})$ where $P$ is a unary predicate symbol. There exists $\mc{C} \succ \mc{B}$ such that $P(\mc{C}) = P(\mc{B})$.
\end{theorem}
\begin{proof}
    Long! Uses a theorem of Ehrenfeucht. Check Reference 8.
\end{proof}

\end{document}