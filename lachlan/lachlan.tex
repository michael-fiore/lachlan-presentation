\documentclass{article}

% Basic document packages.
\usepackage{amsmath,amssymb,amsthm, amsopn}
\usepackage[margin=1.5in]{geometry}

% Font and spacing.
\usepackage{mathpazo}
\usepackage{setspace}
\setstretch{1.25}

% References. I liked the purple color.
\usepackage{xcolor}
\usepackage{hyperref}
\hypersetup{
    colorlinks=true,
    linkcolor=purple
}

% Better arrows for vector notation.
\usepackage{esvect}

% Shortcuts for sets.
\newcommand\N{\mathbb{N}}
\newcommand\Z{\mathbb{Z}}
\newcommand\Q{\mathbb{Q}}
\newcommand\R{\mathbb{R}}
\newcommand\monster{\mathbb{M}}
\DeclareMathOperator{\Diag}{Diag}
\DeclareMathOperator{\Diagel}{Diag_{el}}
\DeclareMathOperator{\Th}{Th}
\DeclareMathOperator{\Tr}{Tr}
\DeclareMathOperator{\RM}{RM}
\DeclareMathOperator{\rank}{rank}
\DeclareMathOperator{\twoR}{R}
\let\mc\mathcal

% Basic environments for theorems et al.
\usepackage{thmtools}
\declaretheoremstyle[bodyfont=\normalfont, notefont=\normalfont\itshape]{std}
\declaretheorem[style=std]{proposition}
\declaretheorem[style=std]{definition}
\declaretheorem[style=std]{notation}
\declaretheorem[style=std]{lemma}
\declaretheorem[style=std]{corollary}
\declaretheorem[style=std]{theorem}
\declaretheorem[style=std]{remark}
\declaretheorem[style=std, numbered=no]{claim}

% Enumeration settings.
\usepackage{enumitem}
\setlist[enumerate,1]{
  label= (\upshape\roman*),
  ref= (\upshape\roman*),
  topsep= 1ex,
  parsep= 0pt,
}

% Boxing text in case you like that.
\usepackage[]{mdframed}
\mdfsetup{linewidth=0.9pt}

% Header for every page.
\usepackage{fancyhdr}
\pagestyle{fancy}
\setlength{\headheight}{15pt}
\renewcommand{\headrulewidth}{0.9pt}

% I hate how paragraphs work in TeX by default.
\usepackage{parskip}
\setlength{\parindent}{15pt}

\begin{document}

\chead{\textbf{Michael Fiore, \'Alvaro Ramos}}

\begin{notation}[Preliminaries] Let $\mc{L}$ be a countable language and let $T$ be a stable theory in $\mc{L}$ which permits infinite models. We will also let $\mc{A}$ and $\mc{B}$ denote $\mc{L}$-structures with universes $A$ and $B$ respectively. When relevant, let $\mc{A} \vDash T$. We will also be using the monster model $\monster \succeq \mc{A}$ for many of the following results.
    \begin{itemize}
        \item Let $S$ be the set of all $\mc{L}$-formulas. Throughout this write-up, we will let $\Delta$ be a set of $\mc{L}$-formulas, so $\Delta \subseteq S$.
        \item Let $\Sigma_n$ be the set of all $\mc{L}$-formulas with at most one free variable $x$.
        \item For any set $X$, let $\mc{L}_X = \mc{L} \cup \{ c_x : x \in X \}$, where we augment $\mc{L}$ with constant symbols for every element of $X$.
        \item Let $S(X)$ be the set of all $\mc{L}_X$ formulas.
        \item Let $\Sigma_1(X)$ be defined similarly.
        \item If $X \subseteq A$, we define $\Th(\mc{A}, X)$ to be the theory of $\mc{A}$ interpreted as an $\mc{L}_X$-structure. In particular, $\Th(\mc{A}, A)$ is the elementary diagram of $\mc{A}$.
        \item An \textit{$A$-instance} of a formula $\varphi$ is any formula obtained from $\varphi$ by substituting $c_a \in \mc{L}_A$ for each variable.
    \end{itemize}
\end{notation}

\begin{definition}[Boolean closure]
    Let $\Delta$ be a set of $\mc{L}$-formulas and let $X$ be a set. We define $\Delta^b(X)$ to be the set of formulas in $\Sigma_1(X)$ obtained from formulas in $\Delta$ using conjunction, disjunction, negation, and substituting instances of $c_x$ for a variable. Trivially, $S^b(A) = S(A)$.
\end{definition}

\begin{definition}[Partitioning]
   Let $\varphi \in \Sigma_1(B)$ and $\Gamma = \{ \psi_1, \ldots, \psi_n \} \subset \Sigma_1(B)$. We say that $\Gamma$ \textit{partitions} $\varphi$ if we have the following.
   \begin{alignat*}{3}
        \mc{B} &\vDash \forall x (\psi_1(x) \lor \cdots \lor \psi_n(x) \lor \neg \varphi(x)) \\
        \mc{B} &\vDash \forall x ((\psi_i(x) \land \psi_j(x)) \to \neg \varphi(x)) &\text{ for all } i \neq j \\
        \mc{B} &\vDash \exists x (\varphi(x) \land \psi_i(x)) &\text{ for all } i
   \end{alignat*}
   In other words, whenever $\varphi(x)$ is true in $\mc{B}$, exactly one of the $\psi_i(x)$'s is true.
\end{definition}

\begin{definition}[Generalized rank and degree]
    Let $\Delta \subseteq A$. In order to define a generalized version of Morley rank and degree, we will need to define two sets $S^{\alpha}(\mc{A}, \Delta)$ and $\Tr^{\alpha}(\mc{A}, \Delta)$ for any ordinal $\alpha$ inductively.
    \begin{itemize}
        \item Let $S^{0}(\mc{A}, \Delta) = \{ \varphi \in \Sigma_1(A) : \mc{A} \vDash \exists x \varphi(x) \}$, i.e. the set of all $\mc{L}$-formulas that have witnesses in $\mc{A}$.

        \item If $S^{\alpha}(\mc{A}, \Delta)$ is already defined, we may define $\Tr^{\alpha}(\mc{A}, \Delta)$ as follows. Given a formula $\varphi$, we say that $\varphi \in \Tr^{\alpha}(\mc{A}, \Delta)$ if and only if there exists a finite $k$ such that for all $\mc{B} \succeq \mc{A}$ and any finite $\Gamma \subset \Delta^{b}(B)$ partitioning $\varphi$, there are no more than $k$ formulas $\psi$ in $\Gamma$ such that $(\varphi \land \psi) \in S^{\alpha}(\mc{B}, \Delta)$.
        
        \item Let $S^{\alpha+1}(\mc{A}, \Delta) = S^{\alpha}(\mc{A}, \Delta) \setminus \Tr^{\alpha}(\mc{A}, \Delta)$.

        \item For $\alpha$ limit, let $S^{\alpha}(\mc{A}, \Delta) = \bigcap_{\beta < \alpha} S^{\beta}(\mc{A}, \Delta)$.

        \item The \textit{$\Delta$-rank} of $\varphi$ in $\mc{A}$ is the least (and unique) ordinal $\alpha$ such that $\varphi \in \Tr^{\alpha}(\mc{A}, \Delta)$. In general we denote this as $\rank_{\mc{A}, \Delta}(\varphi)$.
        
        \item The \textit{$\Delta$-degree} of $\varphi$ in $\mc{A}$ is the least number $k$ witnessing that $\varphi \in \Tr^{\alpha}(\mc{A}, \Delta)$. In general we denote this as $\deg_{\mc{A}, \Delta}(\varphi)$.
        
        \item When $\mc{A}$ or $\Delta$ are unambiguous, we may omit them when writing $\rank$ and $\deg$.
        
        \item For a set of formulas $\Gamma$ (often when $\Gamma$ is a type), we define the $\Delta$-rank of $\Gamma$ to be $\inf \{ \rank(\varphi) : \Gamma \vDash \varphi \}$.
    \end{itemize}
\end{definition}

\begin{proposition}[$\Delta$-rank generalizes Morley rank]\label{morley}
    Let $p$ be a 1-type in $\Th(\mc{A}, A)$. The Morley rank of $p$ is the least $p$ such that $p \cap \Tr^{\alpha}(\mc{A}, S) \neq \varnothing$. The Morley degree of $p$ is the minimum of the $S$-degrees of the formulas in $p \cap \Tr^{\alpha}(\mc{A}, S)$.
\end{proposition}

\begin{proof}
    % Replace B with the monster M.
    Let $\mc{B} \succeq \mc{A}$ be an $\aleph_0$-saturated model, so $p$ is realized in $\mc{B}$. To prove the first part, we show by induction on $\alpha$ that for all $\varphi \in p$,
    \[ \RM(\varphi) = \RM^{\mc{B}}(\varphi) \geq \alpha \Leftrightarrow \varphi \in S^{\alpha}(\mc{A}, S). \]
    If $\alpha = 0$, we clearly have by definition that $\mc{A} \vDash \exists x \varphi(x)$ if and only if $\varphi(\mc{A})$ is nonempty. The limit case follows easily by induction.

    As for the successor case, we need to show that $\RM(\varphi) \geq \alpha + 1$ if and only if $\phi$ is in $S^{\alpha}(\mc{A}, S) \setminus \Tr^{\alpha}(\mc{A}, S)$. First suppose that $\RM(\varphi) \geq \alpha + 1$, so there exist infinitely many $\mc{L}_B$-formulas $\psi_1, \psi_2, \ldots$ defining pairwise disjoint subsets of $\varphi(\mc{B})$ such that $\RM(\psi_i) \geq \alpha$ for all $i < \omega$. Let $k$ be any finite number. Then, define the following finite set of formulas $\Gamma \subset S^b(B)$.
    \[ \Gamma = \left\{ \psi_1, \ldots, \psi_{k+1}, \left( \varphi \land \bigwedge_{i=1}^{k+1} \neg \psi_i \right) \right\} \]
    This set clearly partitions $\varphi$. We can see that for at least $k+1$ formulas $\psi_i \in \Gamma$, $\psi_i \land \varphi = \psi_i$ has rank $\alpha$, which by induction means $\psi_i \in S^{\alpha}(\mc{B}, S)$. Thus, we conclude that $\varphi$ cannot be in $\Tr^{\alpha}(\mc{A}, S)$.

    Now suppose that $\RM(\varphi) = \alpha$. Then, we can see that the Morley degree of $\varphi$ as an $\mc{L}_B$-formula is a $k$ witnessing that $\varphi \in \Tr^{\alpha}(\mc{A}, S)$. In fact, by the definition of the Morley degree of a type, the second part of the proposition is trivial.
\end{proof}

\begin{definition}[Minimality]
    A formula $\varphi$ is \textit{minimal} in $\mc{A}$ if $\varphi \in \Tr^{\alpha}(\mc{A}, \Delta)$ and there is no $\psi \in \Delta$ with some $A$-instance $\psi^{\prime}$ such that $(\varphi \land \psi^{\prime})$ and $(\varphi \land \neg \psi^{\prime})$ are both in $\Tr^{\alpha}(\mc{A}, \Delta)$.
\end{definition}

\begin{lemma}[Properties of rank]\label{pRank}
    Let $\varphi$, $\varphi_0$, $\varphi_1$, and $\psi$ be formulas in $\Sigma_1(A)$.
    \begin{enumerate}
        \item Suppose $\varphi(\mc{A}) \subseteq \psi(\mc{A})$. Then $\rank_{\mc{A}}(\varphi) \leq \rank_{\mc{A}}(\psi)$ and, if $\rank_{\mc{A}}(\varphi) = \rank_{\mc{A}}(\psi)$, then $\deg_{\mc{A}}(\varphi) \leq \deg_{\mc{A}}(\psi)$.
        \item If $\mc{B} \succeq \mc{A}$, then $\rank_{\mc{A}}(\varphi) = \rank_{\mc{B}}(\varphi)$.
        \item If $(\varphi_0 \lor \varphi_1)(\mc{A}) \supseteq \varphi(\mc{A})$ and $\rank_{\mc{A}}(\varphi) \geq \alpha$, then at least one of $\varphi_0$ and $\varphi_1$ has a $\Delta$-rank of at least $\alpha$. In particular, if $\rank_{\mc{A}}(\varphi) = \alpha$ for any ordinal $\alpha$, then $\deg_{\mc{A}}(\varphi) \geq 1$.
    \end{enumerate}
\end{lemma}
\begin{proof}
    (i) Clearly $\varphi \in S^0(\mc{A}, \Delta)$ implies that $\psi(\mc{A}) \supseteq \varphi(\mc{A}) \neq \varnothing$, so $\psi \in S^0(\mc{A}, \Delta)$. For the successor case, given a set which partitions $\varphi$ with $k$ formulas in $S^{\alpha}(\mc{A}, \Delta)$, we easily also have a set which partitions $\psi$ with $k$ formulas in $S^{\alpha}(\mc{A}, \Delta)$. Thus, if $\varphi \notin \Tr^{\alpha}(\mc{A}, \Delta)$, then $\psi \notin \Tr^{\alpha}(\mc{A}, \Delta)$. The limit case follows easily from induction. Also, if $k$ formulas of rank $\alpha$ partition $\varphi$, then these same formulas partition $\psi$, giving the second part of the statement easily.

    (ii) Since $\mc{A} \vDash \exists x \varphi(x)$ if and only if $\mc{B} \vDash \exists x \varphi(x)$, we immediately have the case of $\alpha = 0$. In fact, since the property of a set partitioning a formula depends only on the satisfaction of finitely many sentences, a set partitioning $\varphi$ in $\mc{A}$ with $k$ formulas in $S^{\alpha}(\mc{A}, \Delta)$ will also be a set partitioning $\varphi$ in $\mc{B}$ with $k$ formulas in $S^{\alpha}(\mc{B}, \Delta)$ and vice versa by induction. The limit case also follows easily.
    
    (iii) We have the following deduction to show that this holds for the case of $\alpha = 0$.
    \[ \mc{A} \vDash \exists x \varphi (x) \Rightarrow \mc{A} \vDash \exists x (\varphi_0 (x) \lor \varphi_1(x)) \Rightarrow \mc{A} \vDash \exists x \varphi_0(x) \lor \exists x \varphi_1(x) \]
    
    Suppose the lemma holds for $\alpha$, suppose $\varphi \in S^{\alpha+1}(\mc{A}, \Delta)$, and let $k \in \omega$. There exists a set $\Gamma$ which partitions $\varphi$ with formulas $\psi_1, \ldots, \psi_{2k} \in \Gamma$ such that each $\psi_i$ is in $S^{\alpha}(\mc{A}, \Delta)$. For each $\psi_i$, at least one of $(\psi_i \land \varphi_0)$ and $(\psi_i \land \varphi_1)$ is in $S^{\alpha}(\mc{A}, \Delta)$ by induction. Since we have $2k$ formulas, for at least one of the $\varphi_j$'s there are at least $k$ formulas of the form $(\psi_i \land \varphi_j)$ in $S^{\alpha}(\mc{A}, \Delta)$. Then we can easily construct a partitioning set showing that $\varphi_j$ cannot have degree $k$. We are able to show that for any $k \in \omega$ one of $\varphi_0$ and $\varphi_1$ is unable to have degree $k$. Thus, for one of the $\varphi_j$'s, $\varphi_j$ fails to have degree $k$ for infinitely many $k$, meaning $\varphi_j \in S^{\alpha+1}(\mc{A}, \Delta)$. 
    
    Finally, for the limit case, assuming the lemma holds for all $\beta < \alpha$, it is true for at least one of the $\varphi_j$'s that there is a cofinal sequence of ordinals in $\alpha$ where for each entry $\beta$, $\varphi_j \in S^{\beta}(\mc{A}, \Delta)$. Thus, $\varphi_j \in S^{\alpha}(\mc{A}, \Delta)$.
\end{proof}

In light of \autoref{pRank}(ii), we can always assume we are working in the monster $\monster$ and omit $\monster$ from our notation of $\rank$ and $\deg$.

\begin{lemma}[Definability of $\Delta$-rank]\label{dRank}
    Let $\Delta$ be finite, $n \in \omega$, and let $\psi$ have $\Delta$-rank $n$ and $\Delta$-degree 1. For each $\varphi \in (\Delta \cup \Delta^{-}) \cap \Sigma_{k+1}$ and for any $\mc{B} \succeq \mc{A}$, the set
    \[ D = \{ \overline{b} \in \mc{B}^{k} : \rank_{\mc{B}}(\psi(x) \land \varphi(x, \overline{b})) = n \} \]
    is definable over $A$.
\end{lemma}
\begin{proof}
    We may first suppose that $\mc{B}$ is the monster model $\monster$. We assume that each $\varphi \in \Delta$ is stable since the theory $T$ is stable. Letting $\varphi$ be any formula in $\Delta$ or $\Delta^{-}$ with $k+1$ free variables, we define the following global $\varphi$-type.
    \[ p = \{ \varphi(x, \overline{b}) : \overline{b} \in D \} \cup \{ \neg\varphi(x, \overline{c}) : \overline{c} \in \monster^{k} \setminus D \} \]
    First, we show that this type is consistent, i.e. that $\Diagel(\mc{A}) \cup p$ is satisfiable. Let $p_0 \subset p$ be finite; we'll write this partial $\varphi$-type as
    \[ \left\{ \varphi(x, \overline{b}_1), \ldots, \varphi(x, \overline{b}_r), \neg\varphi(x, \overline{c}_1), \ldots, \neg\varphi(x, \overline{c}_s) \right\} \]
    where $\overline{b}_1, \ldots, \overline{b}_r \in D$ and $\overline{c}_1, \ldots, \overline{c}_s \in \monster^{k} \setminus D$. We will show that
    \[ \theta_{i}(x) = \psi(x) \land \varphi(x, \overline{b}_1) \land \cdots \land \varphi(x, \overline{b}_i) \]
    has $\Delta$-rank $n$ for every $i \in \{ 1, \ldots, r \}$. It will follow, since $n \geq 0$, that this formula has a witness. First, we have that $\rank(\psi(x) \land \varphi(x, \overline{b}_1)) = n$ by definition. Now, let $i \geq 2$ and suppose $\rank(\theta_{i-1}(x)) = n$. We know that $\deg(\theta_{i-1}(x)) = 1$ by \autoref{pRank}(i) and (iii). Thus, if $\rank(\theta_{i-1}(x) \land \varphi(x, \overline{b}_i)) < n$, then it must be the case that $\rank(\theta_{i-1}(x) \land \neg\varphi(x, \overline{b}_i)) = n$. However, by \autoref{pRank}(i), this means that $\rank(\psi(x) \land \neg\varphi(x, \overline{b}_i)) = n$, contradicting that $\psi$ has $\Delta$-degree 1. So, it can only be the case that $\rank(\theta_{i}(x)) = n$. This along with analogous logic for each $\neg\varphi(x, \overline{c}_i)$ shows that
    \[ \rank_{\monster}\left( \psi(x) \land \bigwedge \varphi(x, \overline{b}_i) \land \bigwedge \neg\varphi(x, \overline{c}_i) \right) = n \]
    In particular, since $n \geq 0$, the formula has a witness in $\monster$, meaning that $p_0$ is satisfiable and so $p$ is realized in $\monster$.

    Since $\varphi$ is stable, the $\varphi$-type $p$ is definable by some formula $\theta(\overline{y})$ with parameters in the monster. This formula also defines $D$. To finish our proof, we only need to show that $D$ is $A$-invariant. Let $\sigma$ be any automorphism on $\monster$ which fixes $A$ pointwise. We must show that for any $\overline{b} \in D$, $\sigma(\overline{b}) \in D$. Since the only names in $\psi$ are names of elements of $A$ and the only names in $\varphi(x, \overline{b})$ are $\overline{b}$, we obtain the following.
    \begin{align*}
        \rank(\sigma(\psi(x) \land \varphi(x, \overline{b}))) &=
        \rank(\sigma(\psi(x)) \land \sigma(\varphi(x, \overline{b}))) \\ &=
        \rank(\psi(x) \land \varphi(x, \sigma(\overline{b}))) = n
    \end{align*}
    Thus, $\sigma(\overline{b}) \in D$, meaning $D$ is $A$-invariant. Thus, $D$ is $A$-definable. Since we are working in the monster model, whatever formula defines $D$ in $\monster$ will also define any $\mc{B}$ extending $\mc{A}$, completing our proof.
\end{proof}

\begin{proposition}[Encoding formulas]\label{code}
    Let $A$ and $B$ be sets with $A$ containing at least two elements. Given finitely many formulas $\varphi_1(x, \overline{b}_1), \ldots, \varphi_n(x, \overline{b}_n)$ with fixed parameters in $B$, there exists a formula $\varphi(x, \overline{b}_1, \ldots, \overline{b}_n, \overline{z})$ such that each $\varphi_i$ is equivalent to an $A$-instance of $\varphi$.
\end{proposition}
\begin{proof}
    We can define $\varphi$ explicitly as the conjunction of the following formulas.
    \begin{alignat*}{3}
        \left( \bigwedge_{i=2}^{n} z_1 \neq z_i \right) &\rightarrow& \varphi_1(x, \overline{y}_1) \\
        (z_1 = z_2) &\rightarrow& \varphi_2(x, \overline{y}_2) \\
        (z_1 = z_3 \land z_1 \neq z_2) &\rightarrow& \varphi_3(x, \overline{y}_3) \\
        & \vdots & \\
        \left( z_1 = z_n \land \left( \bigwedge_{i=2}^{n-1} z_1 \neq z_i \right) \right) &\rightarrow& \varphi_n(x, \overline{y}_n)
    \end{alignat*}
    This formula works like a switch-case statement, checking for the leftmost $z_i$ which is equal to $z_1$, where $2 \leq i \leq n$. If $z_1$ is not equal to any of the $z_i$'s, then we default to evaluating $\varphi_1$. Taking two distinct elements $a_1, a_2 \in A$, we can easily express any of the $\varphi_i$'s as an $A$-instance of $\varphi$.
\end{proof}

\begin{corollary}\label{1Form}
    If $\varphi$ encodes each formula in $\Delta$ as in \autoref{code}, then for any $\mc{L}_A$ formula $\psi$,
    \[ \rank_{\Delta}(\psi) = \rank_{\{ \varphi \}}(\psi) \text{ and} \deg_{\Delta}(\psi) = \deg_{\{ \varphi \}}(\psi). \]
\end{corollary}

\begin{lemma}
    If $\Delta$ is finite then the $\Delta$-rank of any formula is finite.
\end{lemma}
\begin{proof}
    First, we define a version of Shelah 2-rank for a formula $\varphi(x)$ given a finite set of $\mc{L}$-formulas $\Delta$. 
    \begin{itemize}
        \item $\twoR_{\Delta}(\varphi) \geq 0$ if $\varphi$ is consistent.
        \item $\twoR_{\Delta}(\varphi) \geq \alpha + 1$ if, for some $\psi(x,y) \in \Delta$ and $a \in \monster^{y}$, we have both of the following.
        \begin{align*} 
            \twoR_{\Delta}(\varphi(x) \land \psi(x, a)) \geq \alpha
            \twoR_{\Delta}(\varphi(x) \land \neg \psi(x, a)) \geq \alpha
        \end{align*}
        \item For a limit ordinal $\gamma$, $\twoR_{\Delta}(\varphi) \geq \gamma$ if $\twoR_{\Delta}(\varphi) \geq \gamma$ for all $\alpha < \gamma$.
    \end{itemize}
    By dint of \autoref{1Form}, we may assume that $\Delta = \{ \psi(x, y) \}$, where $\psi$ is a stable $\mc{L}$-formula. It should be clear that for any $\varphi$, $\rank(\varphi) \leq \twoR(\varphi)$. Now, we show that $\twoR(\varphi)$ must be finite. Assume for the sake of contradiction that $\twoR(x = x) \geq \omega$. Then, using compactness, we can construct a binary tree with countably many parameters where each branch of the tree is a distinct and satisfiable $\psi$-type. But then $\psi$ isn't stable, contradicting our assumption.
\end{proof}

\begin{lemma}
    If $\Delta$ is finite and $\varphi$ is minimal in $\mc{A}$, then the $\Delta$-degree of $\varphi$ is 1.
\end{lemma}
\begin{proof}
    Using \autoref{1Form}, we can assume that $\Delta = \{ \varphi(x, \overline{y}) \}$ contains one stable $\mc{L}$-formula. Suppose $\psi$ is minimal in $\mc{A}$ and has rank $n$. Then, using logic similar to the proof in \autoref{dRank}, we can see that the following $\varphi$-type is consistent and complete in $\mc{A}$.
    \[ p = \{ {\varphi(x, \overline{a})}^{e} : a \in A \text{ and} \rank( {\varphi(x, \overline{a})}^{e} ) = n \text{ where } e = \pm 1 \} \]
    Note that $\rank(p) = n$. Suppose for the sake of contradiction that $\deg(\varphi) \geq 2$, so we can partition it into two formulas $\psi_0$ and $\psi_1$, each of which being in $\Delta^b(\monster)$ and having degree $1$. Then, we may define the following global $\varphi$-type which extends $p$ for $i = 0,1$.
    \[ q_i = \{ {\varphi(x, \overline{a})}^{e} : a \in A \text{ and} \rank( {\psi_i(x) \land \varphi(x, \overline{a})}^{e} ) = n \text{ where } e = \pm 1 \} \]
    Note that each $q_i$ also has rank $n$ and that $q_i \restriction A = p$. This means that $q_0$ and $q_1$ are both nondividing extensions of $p$. However, a nondividing extension of $p$ must be unique, meaning $q_0 = q_1$, a contradiction. Therefore, $p$ cannot have degree larger than $1$.
\end{proof}

\end{document}